\documentclass[11pt,a4paper]{article}
\usepackage{parskip}
\usepackage{geometry}
\usepackage{enumitem}
\usepackage{hyperref}
\usepackage{titlesec}
\usepackage{graphicx}  % For the picture
\usepackage{wrapfig}   % For wrapping text around picture
\usepackage{fontawesome}  % For social media icons

\geometry{a4paper, margin=1in}
% Formatting for section titles
\titleformat{\section}{\large\bfseries}{}{0em}{}[\titlerule]
% Header Information
\pagestyle{empty}

\begin{document}

% Header with picture
\begin{minipage}{0.7\textwidth}
    {\LARGE \textbf{Dr. Sathyanarayan Rao}} \\
    \vspace{2mm}
    \href{mailto:satraox@gmail.com}{satraox@gmail.com} | \href{http://www.drsrao.com}{www.drsrao.com} \\
    \faYoutube\ \href{https://youtube.com/@ComputeStories}{Compute Stories} | 
    Research Software Engineer \\
    Phenorob Project | Forschungszentrum Jülich, Germany
\end{minipage}
\begin{minipage}{0.3\textwidth}
    \includegraphics[width=\linewidth]{photo.png}  % Placeholder for your picture
\end{minipage}

\vspace{5mm}

% Profile Section
\section*{Profile}
Scientific Programmer \& Software Developer with expertise in computational simulations, data analytics, and agricultural modeling. Kaggle Hobbyist with Master rank and a programmer with extensive experience in high-performance computing and model coupling. Creator of educational content on computational topics through YouTube channel "Compute Stories".

% Education Section
\section*{Education}
\begin{itemize}[leftmargin=*]
    \item \textbf{PhD in Engineering Sciences}, UCLouvain, Belgium \hfill 2016 -- 2020 \\
          Thesis: Computational Modeling of Electrical Signatures of Plant Roots \\
          Advisor: Prof. Mathieu Javaux
    \item \textbf{MS in Optical Physics}, Alabama A\&M University, USA \hfill 2013 -- 2014 \\
          GPA: 4.0/4.0
    \item \textbf{MS in Electrical Engineering}, University of Alabama in Huntsville, USA \hfill 2010 -- 2012 \\
          GPA: 3.9/4.0 \\
          Thesis Advisor: Prof. Nagendra Singh
    \item \textbf{B.Eng in Electronics and Communication}, VTU, India \hfill 2006 -- 2010 \\
          First Class with Distinction
\end{itemize}

% Professional Experience Section
\section*{Professional Experience}
\begin{itemize}[leftmargin=*]
    \item \textbf{Scientific Software Engineer}, Phenorob Project, Forschungszentrum Jülich \hfill 2023 -- Present \\
          - Developed coupling mechanisms for crop models in Fortran, C++, and Python \\
          - Created and maintained \href{http://phenorobdaa.de}{phenorobdaa.de} using Hugo \\
          - Produced educational content through YouTube tutorials on crop modeling tools \\
          - Led monthly project meetings and contributed to book chapters
    
    \item \textbf{Research Associate}, Indian Institute of Science, Bengaluru \hfill 2022 -- 2023 \\
          - Developed ML models for soil moisture estimation using LSTM networks \\
          - Created popular Kaggle notebooks with over 1000 views \\
          - Led field experiments and trained researchers in data collection
    
    \item \textbf{Visiting Researcher}, University of Bonn, Germany \hfill 2017 -- 2019 \\
          - Conducted computational analysis of plant root electrical signatures \\
          - Collaborated on finite element modeling with Prof. Andreas Kemna
\end{itemize}

% Technical Skills Section
\section*{Technical Skills}
\begin{itemize}[leftmargin=*]
    \item \textbf{Programming Languages:} Python, C++, Fortran, MATLAB, JavaScript
    \item \textbf{Scientific Computing:} High Performance Computing, Model Coupling, Data Analytics
    \item \textbf{AI/ML:} TensorFlow, PyTorch, scikit-learn, LSTM, Neural Networks
    \item \textbf{Web Development:} Hugo, HTML, CSS, Tailwind CSS, Jekyll
    \item \textbf{Development Tools:} Git, CI/CD Pipeline, GitHub Actions, Jira
\end{itemize}

% Achievements Section
\section*{Achievements}
\begin{itemize}[leftmargin=*]
    \item \textbf{Kaggle Master:} Ranked 649 of 322,985 users, 10 Silver \& 10 Bronze Medals
    \item \textbf{MATLAB Excellence:} Ranked 164 of 19,325, 70,339 Downloads, 4.40 Rating
    \item \textbf{YouTube Channel:} Creator of "Compute Stories", teaching computational topics
\end{itemize}

% Publications Section
\section*{Publications}
\subsection*{Journal Articles}
\begin{enumerate}[leftmargin=*]
    \item "Imaging plant responses to water deficit using electrical resistivity tomography", \textit{Plant \& Soil}, 2020
    \item "Sensing the electrical properties of roots: A review", \textit{Vadose Zone Journal}, 2020
    \item "Geo-electrical methods for root signatures", PhD thesis, \textit{UCL-Université Catholique de Louvain}, 2020
    \item "Impact of maize roots on soil--root electrical conductivity", \textit{Vadose Zone Journal}, 2019
    \item "Waves in helicon magnetic nozzle plasma", \textit{Physics of Plasma}, 2013
    \item "Current-free double layers in a helicon device", \textit{Physics of Plasma}, 2012
    \item "Plasma turbulence from shear Alfvén waves", \textit{Physics of Plasma}, 2012
\end{enumerate}

\subsection*{Book Chapters}
\begin{enumerate}[leftmargin=*]
    \item "Digital Agricultural Avatar: Integrative Crop Modeling for Agricultural Resilience and Climate Change Adaptation", \textit{Springer}, In Preparation
    \item "Can Language Models Revolutionize Climate Smart Agriculture? Navigating Applications, Challenges, and Strategic Approaches", \textit{Springer}, In Preparation
\end{enumerate}

\subsection*{Conference Presentations}
\begin{enumerate}[leftmargin=*]
    \item "Soil Moisture Workshop, Random Forest for Soil Moisture retrieval", IIT Bombay, 2023
    \item "MALM forward modeling with root structure", Geophysical Research Abstracts, 2019
    \item "Electrical anisotropy as root phenotyping, numerical study", Geophysical Research Abstracts, 2019
    \item "Electrical anisotropy and root system architecture", National Symposium for Applied Biological Sciences, 2019
    \item "Characterization of root electrical properties", 5th International Workshop on Induced Polarization, 2018
    \item "Electrical signature of root systems", AGU Fall Meeting Abstracts, 2018
    \item "Electrical Properties of Soil-Root Continuum", AGU Fall Meeting Abstracts, 2018
    \item "Anisotropy in induced polarization of maize root--soil", International Conference on Terrestrial Systems Research, 2018
    \item "Electrical conduction model in soil-root continuum", 4th International Workshop on Geoelectrical Monitoring, 2017
    \item "Electrical resistivity Tomography for root systems", EGU General Assembly Conference Abstracts, 2017
\end{enumerate}

% Fellowships & Grants
\section*{Fellowships \& Grants}
FNRS Fellowship (2016-2020), NSF Fellowship (2011-2012), DFG Grant TVL-E13 (2015-2016), NASA Funded Project (2011-2012)

\end{document}

\end{document}