\documentclass[11pt,a4paper]{article}
\usepackage{parskip}
\usepackage{geometry}
\usepackage{enumitem}
\usepackage{hyperref}
\geometry{a4paper, margin=1in}
\pagestyle{empty}

\begin{document}

% Header
\begin{center}
    {\huge \textbf{Dr. Sathyanarayan Rao}} \\
    \href{mailto:satraox@gmail.com}{satraox@gmail.com} | 
    \href{http://www.drsrao.com}{www.drsrao.com} | 
    \href{https://www.linkedin.com/in/sathyanarayan-rao}{LinkedIn} \\
    \textbf{Data Scientist | Digital Twin Developer | Computational Scientist} \\
\end{center}

% Summary
\section*{Summary}
Data Scientist with expertise in predictive analytics, model coupling, and scientific computing. 
Specialized in creating digital twin solutions for agricultural and environmental systems.
Strong proficiency in Python, C++, TensorFlow, and multi-language integration (C++/Fortran/Python). 
Kaggle Master with demonstrated ability to deliver data-driven solutions for complex industry challenges.

% Skills
\section*{Technical Skills}
\begin{itemize}[leftmargin=*]
    \item \textbf{Programming:} Python, C++, MATLAB, Fortran
    \item \textbf{Model Integration:} API Development, Model Coupling, Digital Twin Architecture
    \item \textbf{ML Frameworks:} TensorFlow, PyTorch, Scikit-learn, Keras
    \item \textbf{Software Engineering:} Docker, Git, CI/CD, WebSockets, API Integration
    \item \textbf{Web Development:} Hugo, JavaScript, HTML, CSS, PyQt
\end{itemize}

% Experience
\section*{Work Experience}

\textbf{Scientific Software Engineer | Phenorob Project, Forschungszentrum Jülich} \hfill 2023 – 2024  
\begin{itemize}[leftmargin=*]
    \item Architected project website using Hugo framework, enhancing stakeholder engagement and knowledge transfer.
    \item Developed interactive GUI for process-based crop models using PyQt, improving user adoption.
    \item Engineered model coupling solutions between C++ and Fortran codebases for integrated digital twin simulations.
    \item Implemented LSTM models for time-series data forecasting with published code reaching 1000+ users on Kaggle.
\end{itemize}

\textbf{Research Associate | Indian Institute of Science, Bengaluru} \hfill 2022 – 2023  
\begin{itemize}[leftmargin=*]
    \item Designed predictive models for soil moisture forecasting with satellite data integration.
    \item Processed terabytes of remote sensing datasets for climate analytics and agricultural insights.
    \item Created high-engagement Kaggle notebooks and models with 1000+ views and industry application.
    \item Developed interactive Vue.js web application for visualizing agricultural datasets.
\end{itemize}

\newpage

% Projects
\section*{Professional Projects}

\textbf{Digital Agricultural Avatar (Model Coupling, Hugo, JavaScript)}
\begin{itemize}[leftmargin=*]
    \item Led development of digital twin platform integrating multiple crop models for precision agriculture.
    \item Created cross-language integration solutions between C++ plant-scale and Fortran crop-scale models.
    \item Implemented responsive interfaces for visualizing complex agricultural simulations and scenarios.
\end{itemize}

\textbf{Docker Containers for Agricultural Modeling (Docker, DevOps)}
\begin{itemize}[leftmargin=*]
    \item Created containerized solutions for complex scientific software deployment, reducing setup time.
    \item Developed CPlantBox GUI Docker with VNC viewer for 3D plant modeling visualizations.
    \item Built DuMuX-ROSI-Jupyter Docker enabling seamless integration of simulation tools for stakeholders.
\end{itemize}

\textbf{AI \& Predictive Analytics Projects (TensorFlow, Data Pipeline Engineering)}
\begin{itemize}[leftmargin=*]
    \item Developed LSTM and Random Forest models for environmental variable forecasting using satellite data.
    \item Achieved Kaggle Master status (Top 2\%) with 10 Silver \& 12 Bronze medals.
\end{itemize}

\textbf{Real-time Data Processing System (Python, WebSockets, Financial APIs)}
\begin{itemize}[leftmargin=*]
    \item Implemented automated data processing system using real-time streams via WebSockets.
    \item Integrated multiple broker APIs for seamless data access and processing.
    \item Developed technical indicator algorithms for optimized decision-making in time-sensitive applications.
\end{itemize}

\textbf{High-Performance Computing Solutions (FORTRAN, MPI, MATLAB)}
\begin{itemize}[leftmargin=*]
    \item Engineered parallel computing simulations for complex physical systems with published results.
    \item Created scalable computational models for environmental systems.
\end{itemize}

\textbf{Healthcare Analytics Application (Gradio, Python)}
\begin{itemize}[leftmargin=*]
    \item Designed scalable machine learning architecture for healthcare analytics as a consultant.
    \item Developed interactive prototype using Gradio for stakeholder visualization and feedback.
\end{itemize}

\textbf{Educational Content Creation (Final Cut Pro, Video Production)}
\begin{itemize}[leftmargin=*]
    \item Created professional educational content on computational science.
    \item Developed "Compute Stories" youtube channel with technical concepts.
    \item Applied video production skills to effectively communicate complex scientific topics.
\end{itemize}

\newpage
% Education
\section*{Education}
\textbf{PhD in Engineering Sciences}, UCLouvain, Belgium \hfill 2016 -- 2020  
Thesis: Predictive Modeling of Electrical Signatures of Plant Roots  
Advisor: Prof. Mathieu Javaux  

\textbf{MS in Optical Physics}, Alabama A\&M University, USA \hfill 2013 -- 2014  
GPA: 4.0/4.0  

\textbf{MS in Electrical Engineering}, University of Alabama in Huntsville, USA \hfill 2010 -- 2012  
GPA: 3.9/4.0  

\textbf{BE in Electronics and Communication Engineering}, Visvesvaraya Tech University \hfill 2006 -- 2010  
GPA: First Class with Distinction  

% Publications
\section*{Selected Publications}

\subsection*{Journal Articles}
\begin{itemize}[leftmargin=*]
    \item Rao, S., et al. (2020). Imaging plant responses to water deficit using electrical resistivity tomography. \textit{Plant and Soil}, 29 citations.
    \item Rao, S., et al. (2020). Sensing the electrical properties of roots: A review. \textit{Vadose Zone Journal}, 19(1), 71 citations.
    \item Rao, S., et al. (2019). Impact of maize roots on soil–root electrical conductivity: A simulation study. \textit{Vadose Zone Journal}, 18(1), 35 citations.
    \item Singh, N., Rao, S., et al. (2013). Waves generated in the plasma plume of helicon magnetic nozzle. \textit{Physics of Plasmas}, 20(3), 27 citations.
    \item Rao, S., Singh, N. (2012). Numerical simulation of current-free double layers created in a helicon plasma device. \textit{Physics of Plasmas}, 19(9), 39 citations.
    \item Singh, N., Rao, S. (2012). Plasma turbulence driven by transversely large-scale standing shear Alfvén waves. \textit{Physics of Plasmas}, 19(12), 3 citations.
\end{itemize}

\subsection*{Book Chapter}
\begin{itemize}[leftmargin=*]
    \item Rao, S., Ranganath, P. (2025). Climate-Resilient Agriculture: Leveraging Language Models for Mitigation and Adaptation. In \textit{Mitigation and Adaptation Strategies Against Climate Change in Natural Environments}.
\end{itemize}

\subsection*{Conference Proceedings}
\begin{itemize}[leftmargin=*]
    \item Mary, B., Rao, S., et al. (2019). Tree root system mise-à-la-masse (MALM) forward modelling with explicit representation of root structure. \textit{Geophysical Research Abstracts}, 21.
    \item Rao, S., et al. (2019). Investigation of Electrical anisotropy as a root phenotyping parameter: Numerical study with root water uptake. \textit{Geophysical Research Abstracts}, 21.
    \item Rao, S., et al. (2019). Relationship between electrical anisotropy of soil-root continuum and geometrical architecture of root system. \textit{National Symposium for Applied Biological Sciences}.
    \item Ehosioke, S., Garré, S., Kremer, T., Rao, S., et al. (2018). A new method for characterizing the complex electrical properties of root segments.
    \item Nguyen, F., Rao, S., et al. (2018). Modeling Effective Electrical Properties of Soil-Root Continuum to Discriminate Root Traits. \textit{AGU Fall Meeting Abstracts}.
    \item Nguyen, F., Rao, S., et al. (2018). Understanding the electrical signature of root systems at different scales to improve agrogeophysical applications. \textit{AGU Fall Meeting Abstracts}.
    \item Rao, S., et al. (2017). A forward model for electrical conduction in soil-root continuum: a virtual rhizotron study. \textit{4th International Workshop on Geoelectrical Monitoring}.
    \item Rao, S., et al. (2017). Characterizing root system characteristics with Electrical resistivity Tomography: a virtual rhizotron simulation. \textit{EGU General Assembly Conference Abstracts}.
\end{itemize}

\end{document}
